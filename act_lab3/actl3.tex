\documentclass[a4paper, 12pt]{article}
\usepackage[utf8x]{inputenc}
\usepackage[english, russian]{babel}
\usepackage[left=25mm, top=25mm, right=25mm, bottom=25mm]{geometry}
\usepackage{cmap}
\usepackage{indentfirst}
\usepackage{tikz}
\usepackage{float}
\usepackage{amsmath, amsfonts, amssymb}
\usepackage{graphicx}
\usepackage{hyperref}
\usepackage{listings}
\usepackage{caption}
\usepackage{subcaption}
\usepackage{xcolor}
\usepackage{etoolbox}
\usepackage{titlesec}
\pagestyle{plain}
\patchcmd{\tableofcontents}{\contentsname}{\centering\contentsname}{}{}
\titleformat{\section}[block]{\normalfont\large\bfseries\centering}{}{0pt}{}
\titleformat{\subsection}[block]{\normalfont\normalsize\bfseries\centering}{}{0pt}{}
\allowdisplaybreaks
\graphicspath{{src/images/}}
\usetikzlibrary{patterns}
\definecolor{LightGray}{gray}{0.95}
\definecolor{LightGray2}{gray}{0.7}
\lstdefinestyle{code}{
    language=MATLAB, % replace language here
    basicstyle=\footnotesize\ttfamily,
    % numbers=left,
    % numberstyle=\scriptsize\color{gray},
    % stepnumber=1,
    % numbersep=5pt,
    backgroundcolor=\color{LightGray},
    showspaces=false,
    showstringspaces=false,
    showtabs=false,
    tabsize=4,
    captionpos=b,
    breaklines=true,
    breakatwhitespace=false,
    frame=single,
    rulecolor=\color{LightGray2},
    linewidth=\linewidth,
    keywordstyle=\color{blue}\bfseries,
    commentstyle=\color{green!40!black},
    stringstyle=\color{purple},
    escapeinside={\%*}{*)},
    inputencoding=utf8x,
    xleftmargin=0pt,
    framexleftmargin=0pt,
    framexrightmargin=0pt
}
\lstset{style=code}
\hypersetup{
    colorlinks=true,
    linkcolor=blue,
    filecolor=magenta,
    urlcolor=cyan,
    pdftitle={contents setup},
    pdfpagemode=FullScreen,
}


\begin{document}
    \begin{titlepage}

        \begin{center}
        Федеральное государственное автономное образовательное учреждение высшего образования
        «Национальный Исследовательский Университет ИТМО»
        \vfill
        
        \includegraphics[width=0.3\textwidth]{itmo.png} % requires /src/images/itmo.png

        {\large\bf ЛАБОРАТОРНАЯ РАБОТА №3}\\
        {\large\bf ПРЕДМЕТ «ТЕОРИЯ АВТОМАТИЧЕСКОГО УПРАВЛЕНИЯ»}\\
        {\large\bf ТЕМА «РЕГУЛЯТОРЫ С ЗАДАННОЙ СТЕПЕНЬЮ УСТОЙЧИВОСТИ»}\\
        Вариант №2
        \vfill

        \begin{flushright}
            \begin{minipage}{.45\textwidth}
            {
                \hbox{Преподаватель:}
                \hbox{Пашенко А. В.}
                \hbox{}
                \hbox{Выполнил:}
                \hbox{Румянцев А. А.}
                \hbox{}
                \hbox{Факультет: СУиР}
                \hbox{Группа: R3341}
                \hbox{Поток: ТАУ R22 бак 1.1.1}
            }
            \end{minipage}
        \end{flushright}
        \vfill
  
        Санкт-Петербург\\
        2025
        \end{center}
    \end{titlepage}
    
    \tableofcontents

    \newpage
    \section{Задание 1. Синтез регулятора с заданной степенью устойчивости}
    Рассмотрим систему
    $$
    \dot{x}=Ax+Bu,\ A=\begin{bmatrix}
        5 &2 &7\\
        2 &1 &2\\
        -2 &-3 &-4
    \end{bmatrix},\ B=\begin{bmatrix}
        3\\1\\-1
    \end{bmatrix};
    $$


    \subsection{Управляемость и стабилизируемость}
    Найдем собственные числа матрицы $A$ и определим управляемость каждого из них. Программа для
    вычислений в \texttt{MATLAB} представлена на листинге \ref{task1code} в приложении 1
    $$
    \sigma\left( A \right)=\left\{-2,2\pm i\right\}
    $$
    Число $\lambda_1=-2$ асимптотически устойчивое, может быть неуправляемым. Комплексная пара $\lambda_{2,3}$
    имеет положительную действительную часть -- эти собственные числа неустойчивые, нужна управляемость.
    Разложим $A$ в вещественную жорданову форму, найдем вектор $B$ в базисе собственных векторов матрицы $A$
    $$
    A=P_{re}J_{re}P_{re}^{-1}=\begin{bmatrix}
    -1    &0.5   &-1.5\\
    0         &0   &-1\\
    1         &0    &1
    \end{bmatrix}\begin{bmatrix}
    -2     &0     &0\\
     0     &2     &1\\
     0    &-1     &2
    \end{bmatrix}\begin{bmatrix}
    0     &1     &1\\
     2    &-1     &2\\
     0    &-1     &0
    \end{bmatrix},
    $$
    $$
    B_{Jre}=P_{re}^{-1}B=\begin{bmatrix}
        0     &1     &1\\
         2    &-1     &2\\
         0    &-1     &0
        \end{bmatrix}\begin{bmatrix}
            3\\
            1\\
            -1
        \end{bmatrix}=\begin{bmatrix}
        0\\
     3\\
    -1
    \end{bmatrix};
    $$
    Итого имеем
    $$
    J_{re}=\begin{bmatrix}
        -2     &0     &0\\
         0     &2     &1\\
         0    &-1     &2
        \end{bmatrix},\ B_{Jre}=\begin{bmatrix}
            0\\
         3\\
        -1
        \end{bmatrix};
    $$
    Все жордановы клетки относятся к различным собственным числам. Только число $\lambda_1=-2$ неуправляемое,
    так как первый элемент $B_{Jre}$ равен нулю. Остальные собственные числа управляемые. Таким образом,
    система не полностью управляема, стабилизируема.


    \subsection{Степень устойчивости}
    Любой степени устойчивости при помощи регулятора $u=Kx$ добиться не получится, так как система не полностью управляема.
    Степень устойчивости системы $\alpha$ -- самое близкое
    к правой комплексной полуплоскости собственное число матрицы $A$,
    находящееся в левой комплексной полуплоскости. Проверка на близость
    осуществляется через действительную часть собственного числа. Имеем
    $$
    \text{Re}\left\{ \lambda_1=-2 \right\}=-2,
    $$
    $$\text{Re}\left\{ \lambda_{2,3}=2\pm i \right\}=2;
    $$
    Таким образом, степень устойчивости системы $\alpha=2$. Это максимум. Устойчивость в данном случае
    подразумевается экспоненциальная.


    \subsection{Схема моделирования системы, замкнутой регулятором}
    Построим схему моделирования системы $\dot{x}=Ax+Bu$, замкнутой регулятором $u=Kx$, используя \texttt{SIMULINK}
    \begin{figure}[H]
        \centering
        \includegraphics[scale=0.5]{scheme_task1.png}
        \captionsetup{skip=0pt}
        \caption{Схема моделирования системы, замкнутой регулятором}
        \label{fig:scheme_task1}
    \end{figure}


    \subsection{Значения желаемой степени устойчивости}
    Возьмем достаточно отличающиеся достижимые степени устойчивости в диапазоне $0<\alpha\leq2$
    \begin{align*}
        &\alpha_{1}=2,\\
        &\alpha_2=0.1;
    \end{align*}


    \subsection{Синтез регулятора через матричное неравенство типа Ляпунова}
    Для каждого из выбранных значений $\alpha$ синтезируем регулятор, обеспечивающий
    заданную степень устойчивости, при помощи матричного неравенства типа Ляпунова
    $$
    PA^T+AP+2\alpha P+Y^TB^T+BY\preceq0,\ K=YP^{-1};
    $$


    Найдем для $\alpha_{1,2}$ соответствующие матрицы регулятора $K_{1\,\alpha_i}$ \textbf{без ограничений на управление}.
    Пользуемся пакетом \texttt{cvx} для \texttt{MATLAB}. Получаем
    $$
    K_{1\,\alpha_1}=\begin{bmatrix}
        2.5267  &-18.8652    &1.7294
    \end{bmatrix},
    $$
    $$
    K_{1\,\alpha_2}=\begin{bmatrix}
        -2.0955   &-5.8106   &-2.6863
    \end{bmatrix};
    $$


    Найдем для $\alpha_{1,2}$ соответствующие матрицы регулятора $K_{2\,\alpha_i}$ \textbf{совместно с решением задачи минимизации управления}.
    Нам нужно найти минимальное $\mu$, для которого при начальных условиях $x(0)=x_0$
    выполняется $||u(t)||\leq\mu$. Для этого нужно решить задачу выпуклой минимизации:
    \begin{align*}
    &\text{минимизировать }\gamma=\mu^2\\
    &\text{при ограничениях } P\succ0,\ PA^T+AP+2\alpha P+Y^TB^T+BY\prec0,\\
    &\begin{bmatrix}
        P &x_0\\
        x_0^T &1
    \end{bmatrix}\succ0,\ \begin{bmatrix}
        P &Y^T\\
        Y &\gamma I
    \end{bmatrix};
    \end{align*}
    Зададим начальные условия $$x(0)=\begin{bmatrix}
        1 \\1 \\1
    \end{bmatrix}$$
    Реализация представлена в \texttt{MATLAB}, для решения используется \texttt{cvx}. Получаем
    $$
    K_{2\,\alpha_1}=\begin{bmatrix}
        1.6000  &-11.2000    &1.6000
    \end{bmatrix},\ \mu_1=8.0090,
    $$
    $$
    K_{2\,\alpha_2}=\begin{bmatrix}
        -0.7565   &-2.6884   &-0.7552
    \end{bmatrix},\ \mu_2=4.2015;
    $$


    Определим собственные числа матриц замкнутых систем $\left( A+BK_{j\,\alpha_i} \right)$
    и сравним с желаемой степенью устойчивости
    \begin{align*}
    &\sigma\left( A+BK_{1\,\alpha_1} \right)=\left\{ -2, -4.5072\pm3.2145i \right\},\\
    &\sigma\left( A+BK_{1\,\alpha_2} \right)=\left\{ -2,-4.5455,-0.8653 \right\},\\
    &\sigma\left( A+BK_{2\,\alpha_1} \right)=\left\{ -2, -2.0000\pm4.1231i \right\},\\
    &\sigma\left( A+BK_{2\,\alpha_2} \right)=\left\{ -2,-0.1013\pm2.3259i \right\};
    \end{align*}
    Для $\alpha_1=2$ собственные числа при регуляторе $K_{2\,\alpha_1}$ получились более точными,
    чем при регуляторе $K_{1\,\alpha_1}$. То есть управление будет именно таким, каким мы его хотели $\left( \text{Re}\left\{ \lambda_i \right\}=-\alpha_1 \right)$.
    На графике увидим плавное поведение системы, стабилизирующееся к нулю. Для $\alpha_2=0.1$ ситуация аналогичная
    -- при $K_{2\,\alpha_2}$ действительная часть комплексной пары почти достигла желаемого ограничения на степень устойчивости.
    При $K_{2\,\alpha_2}$ результат более хаотичный. Также в каждом спектре наблюдаем неуправляемое число $-2$, что подтверждает
    корректность расчетов.


    \subsection{Компьютерное моделирование}
    Выполним компьютерное моделирование для всех замкнутых систем, используя схему \texttt{SIMULINK},
    представленную на рис. \ref{fig:scheme_task1}. Построим графики $u(t),x(t)$ при начальных условиях
    $$x(0)=\begin{bmatrix}
        1 \\1 \\1
    \end{bmatrix}$$
    Результаты представлены на рис. \ref{fig:1task_K1a1_u}--\ref{fig:1task_K2a2_x} со следующей страницы
    \newpage
    \vspace*{0.01mm}
    \begin{figure}[H]
        \centering
        \includegraphics{1task_K1a1_u.png}
        \captionsetup{skip=0pt}
        \caption{График $u(t)$ для $\alpha_1=2$ при $K_{1\,\alpha_1}$}
        \label{fig:1task_K1a1_u}
    \end{figure}
    \begin{figure}[H]
        \centering
        \includegraphics{1task_K1a1_x.png}
        \captionsetup{skip=0pt}
        \caption{График $x(t)$ для $\alpha_1=2$ при $K_{1\,\alpha_1}$}
        \label{fig:1task_K1a1_x}
    \end{figure}
    \newpage
    \vspace*{0.01mm}
    \begin{figure}[H]
        \centering
        \includegraphics{1task_K2a1_u.png}
        \captionsetup{skip=0pt}
        \caption{График $u(t)$ для $\alpha_1=2$ при $K_{2\,\alpha_1},\ \mu_1=8.0090$}
        \label{fig:1task_K2a1_u}
    \end{figure}
    \begin{figure}[H]
        \centering
        \includegraphics{1task_K2a1_x.png}
        \captionsetup{skip=0pt}
        \caption{График $x(t)$ для $\alpha_1=2$ при $K_{2\,\alpha_1},\ \mu_1=8.0090$}
        \label{fig:1task_K2a1_x}
    \end{figure}
    \newpage
    \vspace*{0.01mm}
    \begin{figure}[H]
        \centering
        \includegraphics{1task_K1a2_u.png}
        \captionsetup{skip=0pt}
        \caption{График $u(t)$ для $\alpha_2=0.1$ при $K_{1\,\alpha_2}$}
        \label{fig:1task_K1a2_u}
    \end{figure}
    \begin{figure}[H]
        \centering
        \includegraphics{1task_K1a2_x.png}
        \captionsetup{skip=0pt}
        \caption{График $x(t)$ для $\alpha_2=0.1$ при $K_{1\,\alpha_2}$}
        \label{fig:1task_K1a2_x}
    \end{figure}
    \newpage
    \vspace*{0.01mm}
    \begin{figure}[H]
        \centering
        \includegraphics{1task_K2a2_u.png}
        \captionsetup{skip=0pt}
        \caption{График $u(t)$ для $\alpha_2=0.1$ при $K_{2\,\alpha_2},\ \mu_2=4.2015$}
        \label{fig:1task_K2a2_u}
    \end{figure}
    \begin{figure}[H]
        \centering
        \includegraphics{1task_K2a2_x.png}
        \captionsetup{skip=0pt}
        \caption{График $x(t)$ для $\alpha_2=0.1$ при $K_{2\,\alpha_2},\ \mu_2=4.2015$}
        \label{fig:1task_K2a2_x}
    \end{figure}


    \subsection{Сопоставление результатов}
    На рис. \ref{fig:1task_K2a1_u}, \ref{fig:1task_K2a2_u} видим, что систему удается стабилизировать при помощи минимального управления,
    однако на это уходит больше времени, что наблюдается при сравнении поведения систем на рис. \ref{fig:1task_K2a2_x}, \ref{fig:1task_K1a2_x}.
    В случае с рис. \ref{fig:1task_K2a1_x}, \ref{fig:1task_K1a1_x} это менее заметно. В общем результаты без ограничения на управление более гладкие
    и спокойные, но требуют больше управления.


    \subsection{Синтез регулятора через матричное уравнение типа Риккати}
    Для каждого $\alpha$ синтезируем регулятор при помощи матричного уравнения типа Риккати при $\nu=2\text{ и }R=1$
    $$
    A^TP+PA+Q-\nu PBR^{-1}B^TP+2\alpha P=0,\,K=-R^{-1}B^TP;
    $$
    Пользуемся \texttt{MATLAB}. Найдем матрицы регулятора $K_{3\,\alpha_i}$ при $Q=I$
    $$
    K_{3\,\alpha_1}=\begin{bmatrix}
        2.1164  &-13.4942    &1.6777
    \end{bmatrix},
    $$
    $$
    K_{3\,\alpha_2}=\begin{bmatrix}
        -0.8455   &-3.3716   &-0.5697
    \end{bmatrix};
    $$
    Найдем матрицы регулятора $K_{4\,\alpha_i}$ при $Q=0$
    $$
    K_{4\,\alpha_1}=\begin{bmatrix}
        1.6000  &-11.2000    &1.6000
    \end{bmatrix},
    $$
    $$
    K_{4\,\alpha_2}=\begin{bmatrix}
        -0.7560   &-2.6880   &-0.7560
    \end{bmatrix};
    $$
    Определим собственные числа замкнутых систем $\left( A+BK_{j\,\alpha_i} \right)$
    \begin{align*}
    \sigma\left( A+BK_{3\,\alpha_1} \right)=\left\{ -2, -2.4114\pm4.3116i \right\},\\
    \sigma\left( A+BK_{3\,\alpha_2} \right)=\left\{ -2, -0.6692\pm2.3797i \right\},\\
    \sigma\left( A+BK_{4\,\alpha_1} \right)=\left\{ -2, -2.0000\pm4.1231i \right\},\\
    \sigma\left( A+BK_{4\,\alpha_2} \right)=\left\{ -2, -0.1000\pm2.3259i \right\};
    \end{align*}
    В каждом спектре наблюдаем неуправляемое собственное число $-2$ -- это верно. При $Q=0$ желаемая
    степень устойчивости была достигнута -- действительные части собственных чисел совпадают с соответствующими
    $\alpha_i$. При $Q=I$ результат менее точный, чем при $Q=0$, однако собственные числа ближе к желаемой степени устойчивости в сравнении с результатами для регуляторов $K_{1\,\alpha_i}$.
    Спектр $A+BK_{4\,\alpha_1}$ полностью совпадает с результатом для $K_{2\,\alpha_1}$, а в случае с $\alpha_2$ -- почти полностью.
    В общем решение через матричное уравнение типа Риккати дает более точные результаты.


    \subsection{Компьютерное моделирование для дополнительного пункта}
    Для замкнутых систем $A+BK_{j\,\alpha_i}$ выполним компьютерное моделирование -- построим
    графики $u(t),x(t)$ при начальных условиях $$x(0)=\begin{bmatrix}
        1\\1\\1
    \end{bmatrix}$$
    Для $A+BK_{4\,\alpha_1}$ графики строить избыточно -- результаты полностью совпали с результатом для $A+BK_{2\,\alpha_1}$.
    Достаточно посмотреть на рис. \ref{fig:1task_K2a1_u}, \ref{fig:1task_K2a1_x}.


    Далее расположены графики $u(t),x(t)$, смоделированные по схеме, представленной на рис. \ref{fig:scheme_task1}
    \newpage
    \vspace*{0.01mm}
    \begin{figure}[H]
        \centering
        \includegraphics{1task_K3a1_u.png}
        \captionsetup{skip=0pt}
        \caption{График $u(t)$ для $\alpha_1=2$ при $K_{3\,\alpha_1}$}
        \label{fig:1task_K3a1_u}
    \end{figure}
    \begin{figure}[H]
        \centering
        \includegraphics{1task_K3a1_x.png}
        \captionsetup{skip=0pt}
        \caption{График $x(t)$ для $\alpha_1=2$ при $K_{3\,\alpha_1}$}
        \label{fig:1task_K3a1_x}
    \end{figure}
    \newpage
    \vspace*{0.01mm}
    \begin{figure}[H]
        \centering
        \includegraphics{1task_K3a2_u.png}
        \captionsetup{skip=0pt}
        \caption{График $u(t)$ для $\alpha_2=0.1$ при $K_{3\,\alpha_2}$}
        \label{fig:1task_K3a2_u}
    \end{figure}
    \begin{figure}[H]
        \centering
        \includegraphics{1task_K3a2_x.png}
        \captionsetup{skip=0pt}
        \caption{График $x(t)$ для $\alpha_1=0.1$ при $K_{3\,\alpha_2}$}
        \label{fig:1task_K3a2_x}
    \end{figure}
    \newpage
    \vspace*{0.01mm}
    \begin{figure}[H]
        \centering
        \includegraphics{1task_K4a2_u.png}
        \captionsetup{skip=0pt}
        \caption{График $u(t)$ для $\alpha_2=0.1$ при $K_{4\,\alpha_2}$}
        \label{fig:1task_K4a2_u}
    \end{figure}
    \begin{figure}[H]
        \centering
        \includegraphics{1task_K4a2_x.png}
        \captionsetup{skip=0pt}
        \caption{График $x(t)$ для $\alpha_1=0.1$ при $K_{4\,\alpha_2}$}
        \label{fig:1task_K4a2_x}
    \end{figure}


    \subsection{Сопоставление результатов для дополнительного пункта}
    Видим, что в случаях с $K_4$ система приобретает больше осцилляций,
    чем с $K_3$ (сравн. рис. \ref{fig:1task_K2a1_x}, \ref{fig:1task_K3a1_x} и \ref{fig:1task_K4a2_x}, \ref{fig:1task_K3a2_x}).
    Время схождения системы к нулю быстрее с $K_3$, однако с $K_4$ требуется меньше управления. В общем графики
    почти совпадают с результатами для $K_{1,2}$ (в случае $K_{4\,\alpha_1}$ результаты полностью совпали с $K_{2\,\alpha_1}$). Таким образом,
    можно предположить, что синтез регулятора через матричное уравнение Риккати почти решает задачу минимизации управления.


    \subsection{Вывод}
    В данном задании был исследован синтез регулятора через матричное неравенство типа Ляпунова и
    матричное уравнение типа Риккати. Были получены графики, подтверждающие корректность расчетов и рассуждений.
    Удалось получить желаемую степень устойчивости с помощью неограниченного и минимального управлений. Результат решения через Риккати
    напоминает результат решения задачи минимизации управления.


    \section{Общий вывод по работе}
    ...


    \section{Приложения}
    \subsection{Приложение 1}
    \begin{lstlisting}[label=task1code, caption={Программа для задания 1}]
    % plant parameters
    A = [5 2 7; 2 1 2; -2 -3 -4];
    B = [3; 1; -1];

    % A matrix eigenvalues
    A_e = eig(A)

    % Jordan matrix
    [P, J] = jordan(A);
    Pre(:,1) = P(:,1);
    Pre(:,2) = imag(P(:,2));
    Pre(:,3) = real(P(:,3))
    Pre_inv = Pre^-1
    J_re = Pre_inv * A * Pre
    B_jre = Pre_inv * B

    % Desired decay rate
    a1 = 2;
    a2 = 0.1;

    % solving LMI no restrictions on control
    cvx_begin sdp
    % a1
    variable P1(3,3) symmetric
    variable Y1(1,3)
    P1 > 0.0001*eye(3);
    P1*A' + A*P1 + 2*a1*P1 + Y1'*B'+ B*Y1 <= 0;
    cvx_end

    cvx_begin sdp
    % a2
    variable P2(3,3) symmetric
    variable Y2(1,3)
    P2 > 0.0001*eye(3);
    P2*A' + A*P2 + 2*a2*P2 + Y2'*B'+ B*Y2 <= 0;
    cvx_end

    K1_a1 = Y1*inv(P1)
    K1_a2 = Y2*inv(P2)

    % A+BK1_ai eigenvalues
    ABK1_a1 = A+B*K1_a1;
    ABK1_a2 = A+B*K1_a2;
    eig(ABK1_a1)
    eig(ABK1_a2)

    % solving LMI with control constraint
    x0 = [1; 1; 1];

    % a1
    cvx_begin sdp
    variable P12(3,3) symmetric
    variable Y12(1,3)
    variable mumu_a1
    minimize mumu_a1
    P12 > 0.0001*eye(3);
    P12*A' + A*P12 + 2*a1*P12 + Y12'*B'+ B*Y12 <= 0;
    [P12 x0;
    x0' 1] > 0;
    [P12 Y12';
    Y12 mumu_a1] > 0;
    cvx_end

    cvx_begin sdp
    % a2
    variable P22(3,3) symmetric
    variable Y22(1,3)
    variable mumu_a2
    minimize mumu_a2
    P22 > 0.0001*eye(3);
    P22*A' + A*P22 + 2*a2*P22 + Y22'*B'+ B*Y22 <= 0;
    [P22 x0;
    x0' 1] > 0;
    [P22 Y22';
    Y22 mumu_a2] > 0;
    cvx_end

    mu_a1 = sqrt(mumu_a1)
    mu_a2 = sqrt(mumu_a2)

    K2_a1 = Y12*inv(P12)
    K2_a2 = Y22*inv(P22)

    % A+BK2_ai eigenvalues
    ABK2_a1 = A+B*K2_a1;
    ABK2_a2 = A+B*K2_a2;
    eig(ABK2_a1)
    eig(ABK2_a2)

    % solving Riccati
    Q1 = eye(3);
    v = 2;
    R = 1;

    % a1
    Aa1 = A + eye(3) * (a1-0.0000000001);
    [P,K,e]=icare(Aa1,sqrt(2)*B,Q1,R);
    K3_a1=-inv(R)*B'*P
    e=eig(A+B*K3_a1)

    % a2
    Aa2 = A + eye(3) * a2;
    [P,K,e]=icare(Aa2,sqrt(2)*B,Q1,R);
    K3_a2=-inv(R)*B'*P
    e=eig(A+B*K3_a2)

    Q2 = 0;
    % a1
    Aa12 = A + eye(3) * (a1-0.0000000001);
    [P,K,e]=icare(Aa12,sqrt(2)*B,Q2,R);
    K4_a1=-inv(R)*B'*P
    e=eig(A+B*K4_a1)

    % a2
    Aa22 = A + eye(3) * a2;
    [P,K,e]=icare(Aa22,sqrt(2)*B,Q2,R);
    K4_a2=-inv(R)*B'*P
    e=eig(A+B*K4_a2)
    \end{lstlisting}
\end{document}