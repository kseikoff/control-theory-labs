\documentclass[a4paper, 12pt]{article}
\usepackage[utf8x]{inputenc}
\usepackage[english, russian]{babel}
\usepackage[left=25mm, top=25mm, right=25mm, bottom=25mm]{geometry}
\usepackage{cmap}
\usepackage{indentfirst}
\usepackage{tikz}
\usepackage{float}
\usepackage{amsmath, amsfonts, amssymb}
\usepackage{graphicx}
\usepackage{hyperref}
\usepackage{listings}
\usepackage{caption}
\usepackage{subcaption}
\usepackage{xcolor}
\usepackage{etoolbox}
\usepackage{titlesec}
\pagestyle{plain}
\patchcmd{\tableofcontents}{\contentsname}{\centering\contentsname}{}{}
\titleformat{\section}[block]{\normalfont\large\bfseries\centering}{}{0pt}{}
\titleformat{\subsection}[block]{\normalfont\normalsize\bfseries\centering}{}{0pt}{}
\allowdisplaybreaks
\graphicspath{{src/images/}}
\usetikzlibrary{patterns}
\definecolor{LightGray}{gray}{0.95}
\definecolor{LightGray2}{gray}{0.7}
\lstdefinestyle{code}{
    language=MATLAB, % replace language here
    basicstyle=\footnotesize\ttfamily,
    % numbers=left,
    % numberstyle=\scriptsize\color{gray},
    % stepnumber=1,
    % numbersep=5pt,
    backgroundcolor=\color{LightGray},
    showspaces=false,
    showstringspaces=false,
    showtabs=false,
    tabsize=4,
    captionpos=b,
    breaklines=true,
    breakatwhitespace=false,
    frame=single,
    rulecolor=\color{LightGray2},
    linewidth=\linewidth,
    keywordstyle=\color{blue}\bfseries,
    commentstyle=\color{green!40!black},
    stringstyle=\color{purple},
    escapeinside={\%*}{*)},
    inputencoding=utf8x,
    xleftmargin=0pt,
    framexleftmargin=0pt,
    framexrightmargin=0pt
}
\lstset{style=code}
\hypersetup{
    colorlinks=true,
    linkcolor=blue,
    filecolor=magenta,
    urlcolor=cyan,
    pdftitle={contents setup},
    pdfpagemode=FullScreen,
}


\begin{document}
    \begin{titlepage}

        \begin{center}
        Федеральное государственное автономное образовательное учреждение высшего образования
        «Национальный Исследовательский Университет ИТМО»
        \vfill
        
        \includegraphics[width=0.3\textwidth]{itmo.png} % requires /src/images/itmo.png

        {\large\bf ЛАБОРАТОРНАЯ РАБОТА №4}\\
        {\large\bf ПРЕДМЕТ «ТЕОРИЯ АВТОМАТИЧЕСКОГО УПРАВЛЕНИЯ»}\\
        {\large\bf ТЕМА «СЛЕЖЕНИЕ И КОМПЕНСАЦИЯ: ВИРТУАЛЬНЫЙ ВЫХОД»}\\
        Вариант №2
        \vfill

        \begin{flushright}
            \begin{minipage}{.45\textwidth}
            {
                \hbox{Преподаватель:}
                \hbox{Пашенко А. В.}
                \hbox{}
                \hbox{Выполнил:}
                \hbox{Румянцев А. А.}
                \hbox{}
                \hbox{Факультет: СУиР}
                \hbox{Группа: R3341}
                \hbox{Поток: ТАУ R22 бак 1.1.1}
            }
            \end{minipage}
        \end{flushright}
        \vfill
  
        Санкт-Петербург\\
        2025
        \end{center}
    \end{titlepage}
    
    \tableofcontents

    \newpage
    \section{Задание 1. Компенсирующий регулятор по состоянию}
    Рассмотрим систему
    $$
    \dot{x}=Ax+Bu+B_f\omega_f,
    $$
    $$
    A=\begin{bmatrix}
        5 &2 &7\\
        2 &1 &2\\
        -2 &-3 &-4
    \end{bmatrix},\ B=\begin{bmatrix}
        3\\1\\-1
    \end{bmatrix},\ B_f=\begin{bmatrix}
        -4 &0 &0 &-1\\
        0 &0 &0 &0\\
        4 &0 &0 &0
    \end{bmatrix},\ x(0)=\begin{bmatrix}
        0\\0\\0
    \end{bmatrix},
    $$
    генератор внешнего возмущения
    $$
    \dot{\omega}_f=\Gamma\omega_f,\ \Gamma=\begin{bmatrix}
        25 &6 &-20 &11\\
        14 &3 &-10 &4\\
        40 &11 &-31 &17\\
        6 &4 &-4 &3
    \end{bmatrix},\ \omega_f(0)=\begin{bmatrix}
        1\\1\\1\\1
    \end{bmatrix}
    $$
    и виртуальный выход вида
    $$
    z=C_Zx,\ C_Z=\begin{bmatrix}
        -2 &1 &-1
    \end{bmatrix};
    $$

    
    \subsection{Характер внешнего возмущения}
    ...


    \subsection{Схема моделирования системы, замкнутой компенсирующим регулятором}
    ...


    \subsection{Синтез <<фидбек>>-компоненты компенсирующего регулятора}
    ...


    \subsection{Синтез <<фидфорвард>>-компоненты компенсирующего регулятора}
    ...


    \subsection{Компьютерное моделирование разомкнутой системы}
    ...


    \subsection{Компьютерное моделирование системы, замкнутой регулятором только с <<фидбек>>-компонентой}
    ...


    \subsection{Компьютерное моделирование системы, замкнутой компенсирующим регулятором}
    ...


    \subsection{Вывод}
    ...


    \section{Задание 2. Следящий регулятор по состоянию}
    ...


    \section{Общий вывод по работе}
    ...


    \section{Приложения}
    \subsection{Приложение 1}
    \begin{lstlisting}{label=task1, caption={Программа для задания 1}}
        to be done
    \end{lstlisting}
\end{document}