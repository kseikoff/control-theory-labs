\documentclass[a4paper, 12pt]{article}
\usepackage[utf8x]{inputenc}
\usepackage[english, russian]{babel}
\usepackage[left=25mm, top=25mm, right=25mm, bottom=25mm]{geometry}
\usepackage{cmap}
\usepackage{indentfirst}
\usepackage{tikz}
\usepackage{float}
\usepackage{amsmath, amsfonts, amssymb}
\usepackage{graphicx}
\usepackage{hyperref}
\usepackage{listings}
\usepackage{caption}
\usepackage{subcaption}
\usepackage{xcolor}
\usepackage{etoolbox}
\usepackage{titlesec}
\pagestyle{plain}
\patchcmd{\tableofcontents}{\contentsname}{\centering\contentsname}{}{}
\titleformat{\section}[block]{\normalfont\large\bfseries\centering}{}{0pt}{}
\titleformat{\subsection}[block]{\normalfont\normalsize\bfseries\centering}{}{0pt}{}
\allowdisplaybreaks
\graphicspath{{src/images/}}
\usetikzlibrary{patterns}
\definecolor{LightGray}{gray}{0.95}
\definecolor{LightGray2}{gray}{0.7}
\lstdefinestyle{code}{
    language=MATLAB, % replace language here
    basicstyle=\footnotesize\ttfamily,
    % numbers=left,
    % numberstyle=\scriptsize\color{gray},
    % stepnumber=1,
    % numbersep=5pt,
    backgroundcolor=\color{LightGray},
    showspaces=false,
    showstringspaces=false,
    showtabs=false,
    tabsize=4,
    captionpos=b,
    breaklines=true,
    breakatwhitespace=false,
    frame=single,
    rulecolor=\color{LightGray2},
    linewidth=\linewidth,
    keywordstyle=\color{blue}\bfseries,
    commentstyle=\color{green!40!black},
    stringstyle=\color{purple},
    escapeinside={\%*}{*)},
    inputencoding=utf8x,
    xleftmargin=0pt,
    framexleftmargin=0pt,
    framexrightmargin=0pt
}
\lstset{style=code}
\hypersetup{
    colorlinks=true,
    linkcolor=blue,
    filecolor=magenta,
    urlcolor=cyan,
    pdftitle={contents setup},
    pdfpagemode=FullScreen,
}


\begin{document}
    \begin{titlepage}

        \begin{center}
        Федеральное государственное автономное образовательное учреждение высшего образования
        «Национальный Исследовательский Университет ИТМО»
        \vfill
        
        \includegraphics[width=0.3\textwidth]{itmo.png} % requires /src/images/itmo.png

        {\large\bf ЛАБОРАТОРНАЯ РАБОТА №C}\\
        {\large\bf ПРЕДМЕТ «ТЕОРИЯ АВТОМАТИЧЕСКОГО УПРАВЛЕНИЯ»}\\
        {\large\bf ТЕМА «СЛЕЖЕНИЕ И КОМПЕНСАЦИЯ: ФРАНКИС, ДЭВИСОН И НАБЛЮДАТЕЛИ»}\\
        Вариант №2
        \vfill

        \begin{flushright}
            \begin{minipage}{.45\textwidth}
            {
                \hbox{Преподаватель:}
                \hbox{Пашенко А. В.}
                \hbox{}
                \hbox{Выполнил:}
                \hbox{Румянцев А. А.}
                \hbox{}
                \hbox{Факультет: СУиР}
                \hbox{Группа: R3341}
                \hbox{Поток: ТАУ R22 бак 1.1.1}
            }
            \end{minipage}
        \end{flushright}
        \vfill
  
        Санкт-Петербург\\
        2025
        \end{center}
    \end{titlepage}
    
    \tableofcontents

    \newpage
    \section{Задание 1. Слежение и компенсация: матричные уравнения}
    Рассмотрим систему
    $$
    \begin{cases}
        \dot{x}=Ax+Bu+B_ff,\\
        y=Cx+Du+D_ff,
    \end{cases} x(0)=\begin{bmatrix}
        0\\0\\0
    \end{bmatrix},
    $$
    генератор внешнего возмущения
    $$
    \begin{cases}
        \dot{w}=\Gamma_f w_f,\\
        f=Y_fw_f,
    \end{cases} w_f(0)=\begin{bmatrix}
        1\\1\\1\\1\\
    \end{bmatrix}
    $$
    и генератор задающего воздействия
    $$
    \begin{cases}
        \dot{w}_g=\Gamma_gw_g,\\
        g=Y_gw_g,
    \end{cases} w_g(0)
    $$
    при параметрах объекта управления
    $$
    A=\begin{bmatrix}
        5 &2 &7\\
        2 &1 &2\\
        -2 &-3 &-4
    \end{bmatrix},\ B=\begin{bmatrix}
        3\\1\\-1
    \end{bmatrix},\ B_f=\begin{bmatrix}
        -4 &-1\\
        0 &0\\
        4 &0
    \end{bmatrix},\ C=\begin{bmatrix}
        2\\0\\3
    \end{bmatrix}^T,\ D=2,\ D_f=\begin{bmatrix}
        8\\3
    \end{bmatrix}^T
    $$
    и параметрах генератора
    $$
    \Gamma_f=\begin{bmatrix}
        25 &6 &-20 &11\\
        14 &3 &-10 &4\\
        40 &11 &-31 &17\\
        6 &4 &-4 &3
    \end{bmatrix},\ Y_f=\begin{bmatrix}
        8 &-20\\
        2 &-6\\
        -6 &16\\
        4 &-9
    \end{bmatrix}^T,\ g(t)=4\sin\left( t \right)-1;
    $$
    Программа для задания 1 находится в приложении А на листинге \ref{task1}


    \subsection{Характер внешнего возмущения}
    Найдем собственные числа матрицы $\Gamma_f$,
    чтобы определить характер внешнего возмущения
    $$
    \sigma\left( \Gamma_f \right)=\left\{ \pm i,\pm3i \right\}
    $$
    Спектр состоит только из мнимых чисел. Характер возмущения --
    гармоники без роста и затухания амплитуды с течением времени.


    \subsection{Генератор задающего воздействия}
    $$
    \Gamma_g=\begin{bmatrix}
        0 &1 &0\\
        -1 &0 &0\\
        0 &0 &0
    \end{bmatrix},\ Y_g=\begin{bmatrix}
        4 &0 &-1
    \end{bmatrix},\ w_g(0)=\begin{bmatrix}
        0\\1\\1
    \end{bmatrix};
    $$


    \subsection{Схема моделирования системы}
    ...


    \subsection{Синтез компоненты обратной связи}
    Исследуем систему на стабилизируемость
    $$
    \sigma\left( A \right)=\left\{ -2,2\pm i \right\},\ A_{J_re}=\begin{bmatrix}
        -2 &0 &0\\
        0 &2 &1\\
        0 &-1 &2
    \end{bmatrix},\ B_{J_re}=\begin{bmatrix}
        0\\3\\-1
    \end{bmatrix};
    $$
    Система не полностью управляема, стабилизируема. Максимальная степень
    устойчивости $\alpha=2$.


    Синтезируем компоненту обратной связи $K$ с помощью матричного уравнения
    типа Риккати
    $$
    A^TP+PA+Q-\nu PBR^{-1}B^TP+2\alpha P=0,\,K=-R^{-1}B^TP;
    $$
    при $Q=I,\nu=2,R=1,\alpha=2$. Получаем
    $$
    K=\begin{bmatrix}
        2.1111  &-13.4448    &1.6787
    \end{bmatrix},$$
    $$
    \sigma\left( A+BK \right)=\left\{ -2,-2.3951\pm 4.3138i \right\};
    $$
    Желаемая степень устойчивости достигнута -- регулятор синтезирован корректно.


    \subsection{Общий вид матричных уравнений Франкиса-Дэвисона}
    Матричные уравнения Франкиса-Дэвисона в общем виде представляются системой
    $$
    \begin{cases}
        AP+BK+Y_1=P\Gamma,\\
        CP+DK+Y_2=0;
    \end{cases}
    $$
    Решение относительно $P$ и $K$ для произвольных
    $Y_1$ и $Y_2$ существует, если
    $$
    \text{rank}\begin{bmatrix}
        A-I\lambda_{i\Gamma} &B\\
        C &D
    \end{bmatrix}=\text{число строк}
    $$
    $\lambda_{i\Gamma}$ -- собственные числа $\Gamma$.


    \subsection{Синтез компоненты слежения}
    ...


    \subsection{Синтез компоненты компенсации по входу}
    ...


    \subsection{Компьютерное моделирование}
    ...


    \subsection{Сравнение результатов}
    ...


    \section{Приложение А}
    \begin{lstlisting}[label=task1, caption={Программа для задания 1}]
    % plant parameters
    A=[5 2 7;
       2 1 2;
      -2 -3 -4];
    B=[3;1;-1];
    Bf=[-4 -1;
        0 0;
        4 0];
    C=[2 0 3];
    D=2;
    Df=[8 3];

    Gf=[25 6 -20 11;
        14 3 -10 4;
        40 11 -31 17;
        6 4 -4 3];
    Yf=[8 2 -6 4;
       -20 -6 16 -9];

    Gg = [0 1 0;
         -1 0 0;
          0 0 0];
    Yg=[4 0 -1];
    wg0 = [0;1;1];

    % A eigenvalues
    A_eig = eig(A)

    % Jordan matrix
    [P1, J] = jordan(A);
    Pjre(:,1) = P1(:,1);
    Pjre(:,2) = imag(P1(:,2));
    Pjre(:,3) = real(P1(:,3))
    Pjre_inv = Pjre^-1
    Aj_re = Pjre_inv * A * Pjre
    B_jre = Pjre_inv * B

    % G eigenvalues
    Gf_eig = eig(Gf)
    Gg_eig = eig(Gg)

    % solving Riccati: feedback comp
    Q = eye(3);
    v = 2;
    R = 1;
    a = 2;

    Aa = A + eye(3) * (a-0.00000000000001);
    [Pk,K,e]=icare(Aa,sqrt(v)*B,Q,R);
    K=-inv(R)*B'*Pk
    eK=eig(A+B*K)

    % check Frankis-Davison: Kg
    check_Kg1 = [A-eye(3)*Gg_eig(1) B; C D]
    rank(check_Kg1)

    check_Kg2 = [A-eye(3)*Gg_eig(2) B; C D]
    rank(check_Kg2)

    check_Kg3 = [A-eye(3)*Gg_eig(3) B; C D]
    rank(check_Kg3)

    % solving Frankis-Davison: Kg
    cvx_begin sdp
    variable Pg(3,3)
    variable Kg(1,3)
    Pg*Gg-A*Pg == B*Kg;
    Yg-C*Pg == D*Kg;
    cvx_end

    Pg=Pg
    Kg=Kg

    % check Frankis-Davison: Kf
    check_Kf1 = [A-eye(3)*Gf_eig(1) B; C D]
    rank(check_Kf1)

    check_Kf2 = [A-eye(3)*Gf_eig(2) B; C D]
    rank(check_Kf2)

    check_Kf3 = [A-eye(3)*Gf_eig(3) B; C D]
    rank(check_Kf3)

    check_Kf4 = [A-eye(3)*Gf_eig(4) B; C D]
    rank(check_Kf4)

    % solving Frankis-Davison: Kf
    cvx_begin sdp
    variable Pf(3,4)
    variable Kf(1,4)
    Pf*Gf-A*Pf-Bf*Yf == B*Kf;
    -C*Pf == D*Kf;
    cvx_end

    Pf=Pf
    Kf=Kf
    \end{lstlisting}
\end{document}